% ------------------------------------------------------------------------------
% The abstract should summarize the contents of the paper and should contain at
% least 70 and at most 150 words. It should be written using the \emph{abstract}
% environment. It must not contain references, as it may be used
% without the main article. It is acceptable, although not common, to identify
% work by author, abbreviation or RFC number (For example, "Our algorithm is
% based upon the work by Smith and Wesson."). Avoid use of "in this paper" in
% the abstract. What other paper would you be talking about here? Avoid general
% motivation in the abstract. You do not have to justify the importance of the
% Internet or explain what QoS is. Highlight not just the problem, but also the
% principal results. Many people read abstracts and then decide whether to
% bother with the rest of the paper. Since the abstract will be used by search
% engines, be sure that terms that identify your work are found there. In
% particular, the name of any protocol or system developed and the general area
% ("quality of service", "protocol verification", "service creation
% environment") should be contained in the abstract. Avoid equations and math.
% ------------------------------------------------------------------------------
% ------------------------------------------------------------------------------
% (1) Define the research area (which particular area are we focusing?).
% ------------------------------------------------------------------------------
The integration of the Fifth-generation (5G) networks with
Time-Sensitive Networking (TSN) technology is believably one of the most considerable argument topics in industrial automation nowadays. Companies that operate an intelligent manufacturing system aim to develop an extensive, ubiquitous, and permanent connectivity of the technologies and features of networking to achieve real-time requirements. That would create advantages such as building up smart factories completely connected, enhanced, and compliant with all the necessary industrial technology requirements. Moreover, the fifth-generation (5G) networks attempt to play a significant role in connecting massive wireless devices, which are very useful in factories.
Furthermore, it has a tremendous positive impact on improving smart factories' capabilities, performance, scalability, and compliance.
All essential elements of our research are discussed in the 3GPP release 15 specifications when the 5G service-based architecture (SBA) was reviewed. The fundamental technological components in the 5G core network are the separation of control plane and user plane, service-based interface (SBI), modularization, and network function virtualization. SBA Architecture provides a single API calling interface.  All the network functions (NF) are interconnected via an interface for calling the other NF. 
After Authorization, the network functions (NFs) able to access the other NF using SBI. Network function virtualization (NFV) empowers the network function to be virtualized and deployed on any cloud environment. The support of virtualization in the 5G, the limits between traditional Evolved Packet Core (EPC) network components (MME, SGW, and PGW) will come to exit \cite{5G_Tech_Spec_Group_Ser2018study}.

  
 

 The paper targets on forming and handling a hybrid network comprising Industrial Ethernet/Time Sensitive Networking (TSN) and 5G, as the key communication system representatives of operational technology (OT) and information and communication technology (ICT) industry. It is organized as follows: the subsections below introduce 5G mobile networks, Industrial Ethernet and TSN, section II describes scenarios to combine both communication technologies, section III describes approach to model and configure hybrid networks and their transitions, and section IV concludes the paper.
 
  
The thesis scope is to investigate the integration of 5GC with TSN And implement a Simulation of a 5G Core Network Deployment and Testing with an open-source 5G SA gNB emulator (Rel. 16) gNBSIM. Besides, It studies the opportunity of transmitting data packets from end-to-end (E2E). Wireshark and Iperf apps will be employed to capture Data Packets. They help achieve a comparative analysis and get the measurement of time-sync, Reliability, Latency, and Determinism.   


%%context
% ------------------------------------------------------------------------------
% (2) Define the issue (what issue is getting to get solved?)
% ------------------------------------------------------------------------------******************
  
  
  
%.%%problem
% ------------------------------------------------------------------------------
% (3) Shortcomings of existing solutions.
% ------------------------------------------------------------------------------
%%work
% ------------------------------------------------------------------------------
% (4) Define the own approach.
% ------------------------------------------------------------------------------



%%approach
% ------------------------------------------------------------------------------
% (5) What is the expected/generated scientific surplus value?
% ------------------------------------------------------------------------------
%%result
% ------------------------------------------------------------------------------
% (6) How have we validated our results?
% ------------------------------------------------------------------------------
%%evaluation
% ------------------------------------------------------------------------------
% (7) How could this work be extended?
% ------------------------------------------------------------------------------
%%outlook

\acresetall%
